\documentclass[11pt]{article}
\usepackage[margin=2.5cm]{geometry} %layout
\usepackage{hyperref}  % this is needed for forms and links within the text
 
\hypersetup{
  pdfauthor   = {Marcin Szczodrak},
  pdfkeywords = {Marcin Szczodrak, Standard Fennec Fox Test Suit},
  pdftitle    = {Standard Fennec Fox Test Suit}
}

\pagestyle{myheadings}
\markright{Fennec Fox Standard Test Suite. \hfill 04/18/2012 Rev: 001 \hfill}

 
\begin{document}

%\thispagestyle{empty}

\begin{center}
\huge Fennec Fox Standard Test Suite
\end{center}

\vspace{2cm}

\begin{Form}
\noindent
\TextField[name=name, width=\textwidth]{Name:\\\\}\\\\
\TextField[name=email, width=\textwidth]{Email:\\\\}\\\\
\TextField[name=date, width=\textwidth]{Date:\\\\} \\\\
\TextField[name=platform, width=\textwidth]{Platform:\\\\}
\end{Form}

\vspace{2cm}

\begin{center}
\begin{Form}
\begin{tabular}{| c | c | c |}
        \hline
        No.             & Test Name     &       Pass / Fail \\
        \hline
        \hline
        1               & Blink         &
                                    \CheckBox[bordercolor={0 0 0},height=0.4cm,width=0.4cm, name=t1p]{Pass}
                                    \CheckBox[bordercolor={0 0 0},height=0.4cm,width=0.4cm, name=t1f]{Fail} \\
        \hline
        2               & Printf        &
                                    \CheckBox[bordercolor={0 0 0},height=0.4cm,width=0.4cm, name=t2p]{Pass}
                                    \CheckBox[bordercolor={0 0 0},height=0.4cm,width=0.4cm, name=t2f]{Fail} \\
        \hline
        3               & Network       &
                                    \CheckBox[bordercolor={0 0 0},height=0.4cm,width=0.4cm, name=t3p]{Pass}
                                    \CheckBox[bordercolor={0 0 0},height=0.4cm,width=0.4cm, name=t3f]{Fail} \\
        \hline



        \hline
\end{tabular}
\end{Form}
\end{center}


\vfill




\noindent {\large Procedure:}

The Fennec Fox Standard Test Suite is based on the Swift Fox test programs
checking if a mote's LEDs are blinking, if the serial and printf libraries
can communicate with a mote and PC, and if a network data collection is 
operating between two motes. The test suite is located in the Swift Fox
directory, in \texttt{\$ swiftfox/examples/tests}.

The suite has been tested on a $platform-name$, where
$platform-name$ in one of the following: telosB, intelMote2, Z1, micaz.

\newpage

\begin{center}
{\large No. 1 : Blink} 	
\end{center}
\vspace{1cm}
	\textbf{INPUT:}\newline
	Test Blinks LEDs. \newline
	\texttt{\$ sfc blink} \newline
	\texttt{\$ fennec $platform-name$ install}\newline
	\textbf{OUTPUT:}\newline
	Should see blinking LEDs. 

\vspace{1cm}
\TextField[multiline=true,height=20\baselineskip, width=\textwidth]{Comments\\\\}
\newpage

\begin{center}
{\large No.2 : Printf}	
\end{center}
\vspace{1cm}
	\textbf{INPUT:}\newline
	\texttt{\$ sfc printf} \newline
	\texttt{\$ fennec $platform-name$ install}\newline
	Using terminal on a computer to which a mote is 
	connected to, the raw serial data is read by: \newline
	\texttt{\$ java net.tinyos.tools.Listen -comm serial@/dev/ttyUSB0:115200} \newline
	To see text messages, serial date should be read by: \newline
	\texttt{\$ java net.tinyos.tools.PrintfClient -comm serial@/dev/ttyUSB0:115200} \newline
	\textbf{OUTPUT:}\newline
	For raw data: \newline
	\texttt{\scriptsize 
		serial@/dev/ttyUSB0:115200: resynchronising\newline
		00 FF FF 00 00 1C 00 64 48 65 6C 6C 6F 20 57 6F 72 6C 64 21 0A 00 00 00 00 00 00 00 00 00 00 00 00 00 00 00\newline
		00 FF FF 00 00 1C 00 64 48 65 6C 6C 6F 20 57 6F 72 6C 64 21 0A 00 00 00 00 00 00 00 00 00 00 00 00 00 00 00\newline
		00 FF FF 00 00 1C 00 64 48 65 6C 6C 6F 20 57 6F 72 6C 64 21 0A 00 00 00 00 00 00 00 00 00 00 00 00 00 00 00\newline
		00 FF FF 00 00 1C 00 64 48 65 6C 6C 6F 20 57 6F 72 6C 64 21 0A 00 00 00 00 00 00 00 00 00 00 00 00 00 00 00\newline
	}
	For text data: \newline
	\texttt{\scriptsize 
		Thread[Thread-1,5,main]serial@/dev/ttyUSB0:115200: resynchronising\newline
		Hello World!\newline
		Hello World!\newline
		Hello World!\newline
		Hello World!\newline
	}

\TextField[multiline=true,height=20\baselineskip, width=\textwidth]{Comments\\\\}
\newpage

\begin{center}
{\large No.3 : Network}
\end{center}
\vspace{1cm}
	\texttt{\$ sfc network} \newline
	Install Fennec Fox on the first mote, giving it address 102.\newline
	\texttt{\$ fennec $platform-name$ install, 102}\newline
	Power-on mote with id 102: the mote's LEDs SHOULD NOT blink.\newline
	Power-off mote with id 102.\newline
	\texttt{\$ fennec $platform-name$ install, 101}\newline
	Power-on mote with id 101: the mote's LEDs SHOULD blink.\newline
	Power-on mote with id 102: the mote's LEDs SHOULD blink the same 
	sequence that mote with 101 does.\newline

\vspace{1cm}
\TextField[multiline=true,height=20\baselineskip, width=\textwidth]{Comments\\\\}
\newpage


\end{document}
