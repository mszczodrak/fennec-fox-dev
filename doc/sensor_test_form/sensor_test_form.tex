\documentclass[11pt]{article}
\usepackage[margin=2.5cm]{geometry} %layout
\usepackage{hyperref}  % this is needed for forms and links within the text
 
\hypersetup{
  pdfauthor   = {Marcin Szczodrak},
  pdfkeywords = {Marcin Szczodrak, Fennec Fox Sensors Test Suit},
  pdftitle    = {Fennec Fox Sensors Test Suit}
}

\pagestyle{myheadings}
\markright{Fennec Fox Sensors Test Suite. \hfill 03/17/2012 Rev: 001 \hfill}

 
\begin{document}

%\thispagestyle{empty}

\begin{center}
\huge Fennec Fox Sensors Test Suite
\end{center}

\vspace{2cm}

\begin{Form}
\noindent
\TextField[name=name, width=\textwidth]{Name:\\\\}\\\\
\TextField[name=email, width=\textwidth]{Email:\\\\}\\\\
\TextField[name=date, width=\textwidth]{Date:\\\\} \\\\
\TextField[name=platform, width=\textwidth]{Platform:\\\\}
\end{Form}

\vspace{2cm}

\begin{center}
\begin{Form}
\begin{tabular}{| c | c | c | c |}
        \hline
        No.             & Test Name     & Virtual Sensor	& Pass / Fail \\
        \hline
        \hline
        1               & Temperature         &
                                    \CheckBox[bordercolor={0 0 0},height=0.4cm,width=0.4cm, name=t1v]{Yes}
					&
                                    \CheckBox[bordercolor={0 0 0},height=0.4cm,width=0.4cm, name=t1p]{Pass}
                                    \CheckBox[bordercolor={0 0 0},height=0.4cm,width=0.4cm, name=t1f]{Fail} \\
        \hline
        2               & Humidity        &
                                    \CheckBox[bordercolor={0 0 0},height=0.4cm,width=0.4cm, name=t2v]{Yes}
					&
                                    \CheckBox[bordercolor={0 0 0},height=0.4cm,width=0.4cm, name=t2p]{Pass}
                                    \CheckBox[bordercolor={0 0 0},height=0.4cm,width=0.4cm, name=t2f]{Fail} \\
        \hline
        3               & Light       &
                                    \CheckBox[bordercolor={0 0 0},height=0.4cm,width=0.4cm, name=t3v]{Yes}
					&
                                    \CheckBox[bordercolor={0 0 0},height=0.4cm,width=0.4cm, name=t3p]{Pass}
                                    \CheckBox[bordercolor={0 0 0},height=0.4cm,width=0.4cm, name=t3f]{Fail} \\
        \hline

        4               & Battery       &
                                    \CheckBox[bordercolor={0 0 0},height=0.4cm,width=0.4cm, name=t4v]{Yes}
                                        &
                                    \CheckBox[bordercolor={0 0 0},height=0.4cm,width=0.4cm, name=t4p]{Pass}
                                    \CheckBox[bordercolor={0 0 0},height=0.4cm,width=0.4cm, name=t4f]{Fail} \\
        \hline

        5               & Accelerometer	&
                                    \CheckBox[bordercolor={0 0 0},height=0.4cm,width=0.4cm, name=t5v]{Yes}
                                        &
                                    \CheckBox[bordercolor={0 0 0},height=0.4cm,width=0.4cm, name=t5p]{Pass}
                                    \CheckBox[bordercolor={0 0 0},height=0.4cm,width=0.4cm, name=t5f]{Fail} \\
        \hline

        6               & Magnetic	&
                                    \CheckBox[bordercolor={0 0 0},height=0.4cm,width=0.4cm, name=t6v]{Yes}
                                        &
                                    \CheckBox[bordercolor={0 0 0},height=0.4cm,width=0.4cm, name=t6p]{Pass}
                                    \CheckBox[bordercolor={0 0 0},height=0.4cm,width=0.4cm, name=t6f]{Fail} \\
        \hline

        7               & PIR		&
                                    \CheckBox[bordercolor={0 0 0},height=0.4cm,width=0.4cm, name=t7v]{Yes}
                                        &
                                    \CheckBox[bordercolor={0 0 0},height=0.4cm,width=0.4cm, name=t7p]{Pass}
                                    \CheckBox[bordercolor={0 0 0},height=0.4cm,width=0.4cm, name=t7f]{Fail} \\
        \hline

        8               & Sound      &
                                    \CheckBox[bordercolor={0 0 0},height=0.4cm,width=0.4cm, name=t8v]{Yes}
                                        &
                                    \CheckBox[bordercolor={0 0 0},height=0.4cm,width=0.4cm, name=t8p]{Pass}
                                    \CheckBox[bordercolor={0 0 0},height=0.4cm,width=0.4cm, name=t8f]{Fail} \\
        \hline


        9               & Camera	&
                                    \CheckBox[bordercolor={0 0 0},height=0.4cm,width=0.4cm, name=t9v]{Yes}
                                        &
                                    \CheckBox[bordercolor={0 0 0},height=0.4cm,width=0.4cm, name=t9p]{Pass}
                                    \CheckBox[bordercolor={0 0 0},height=0.4cm,width=0.4cm, name=t9f]{Fail} \\
        \hline

        10               & Vibration	&
                                    \CheckBox[bordercolor={0 0 0},height=0.4cm,width=0.4cm, name=t10v]{Yes}
                                        &
                                    \CheckBox[bordercolor={0 0 0},height=0.4cm,width=0.4cm, name=t10p]{Pass}
                                    \CheckBox[bordercolor={0 0 0},height=0.4cm,width=0.4cm, name=t10f]{Fail} \\
        \hline

        11               & Thin Force	&
                                    \CheckBox[bordercolor={0 0 0},height=0.4cm,width=0.4cm, name=t11v]{Yes}
                                        &
                                    \CheckBox[bordercolor={0 0 0},height=0.4cm,width=0.4cm, name=t11p]{Pass}
                                    \CheckBox[bordercolor={0 0 0},height=0.4cm,width=0.4cm, name=t11f]{Fail} \\
        \hline



        \hline
\end{tabular}
\end{Form}
\end{center}


\vfill




\noindent {\large Procedure:}

The Fennec Fox Sensor Test Suite is based on the Swift Fox test programs
checking if a mote's LEDs are blinking, if the serial and printf libraries
can communicate with a mote and PC, and if a network data collection is 
operating between two motes. The test suite is located in the Swift Fox
directory, in \texttt{\$ swiftfox/examples/tests}

The suite has been tested on a \texttt{platform-name}, where 
\texttt{platform-name} in one of the following: telosB, intelMote2, Z1, micaz.

\newpage

\begin{center}
{\large No. 1 : Temperature} 	
\end{center}
\vspace{1cm}
	Test Temperature Sensor. \newline
	\textbf{INPUT:}\newline
	\texttt{\$ sfc temp} \newline
	\texttt{\$ fennec \texttt{platform-name} install}\newline
	\textbf{OUTPUT:}\newline
	Messages with sensor measurements can be read through USB: \newline
	\texttt{\$ java net.tinyos.tools.PrintfClient -comm serial@/dev/ttyUSB0:115200} \newline
	Sample output:
	\texttt{\scriptsize 
		Thread[Thread-1,5,main]serial@/dev/ttyUSB0:115200: resynchronising\newline
		Temp: 28\newline
		Temp: 27\newline
		Temp: 28\newline
		Temp: 28\newline
	}

\vspace{1cm}
\TextField[multiline=true,height=20\baselineskip, width=\textwidth]{Comments\\\\}
\newpage

\begin{center}
{\large No.2 : Humidity}	
\end{center}
\vspace{1cm}
        Test Humidity Sensor. \newline
        \textbf{INPUT:}\newline
        \texttt{\$ sfc hum} \newline
        \texttt{\$ fennec \texttt{platform-name} install}\newline
        \textbf{OUTPUT:}\newline
        Messages with sensor measurements can be read through USB: \newline
        \texttt{\$ java net.tinyos.tools.PrintfClient -comm serial@/dev/ttyUSB0:115200} \newline
        Sample output:
        \texttt{\scriptsize
                Thread[Thread-1,5,main]serial@/dev/ttyUSB0:115200: resynchronising\newline
                Hum: 44\newline
                Hum: 43\newline
                Hum: 44\newline
                Hum: 43\newline
        }
\vspace{1cm}
\TextField[multiline=true,height=20\baselineskip, width=\textwidth]{Comments\\\\}
\newpage

\begin{center}
{\large No.3 : Light}
\end{center}
\vspace{1cm}
        Test Light Sensor. \newline
        \textbf{INPUT:}\newline
        \texttt{\$ sfc ligt} \newline
        \texttt{\$ fennec \texttt{platform-name} install}\newline
        \textbf{OUTPUT:}\newline
        Messages with sensor measurements can be read through USB: \newline
        \texttt{\$ java net.tinyos.tools.PrintfClient -comm serial@/dev/ttyUSB0:115200} \newline
        Sample output:
        \texttt{\scriptsize
                Thread[Thread-1,5,main]serial@/dev/ttyUSB0:115200: resynchronising\newline
                Ligt: 540\newline
                Ligt: 480\newline
                Ligt: 370\newline
                Ligt: 380\newline
        }
\vspace{1cm}
\TextField[multiline=true,height=20\baselineskip, width=\textwidth]{Comments\\\\}
\newpage


\end{document}
